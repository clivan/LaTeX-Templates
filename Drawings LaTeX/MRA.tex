\documentclass[aspectratio=169,14pt]{beamer}
\usepackage{tikz}
\usepackage{animate}
\usepackage{ifthen}
\newcounter{a}
\setcounter{a}{0}
\usetikzlibrary{decorations.pathmorphing}
\usetikzlibrary{decorations.markings,patterns}
\tikzset{
resorte/.style={decorate,draw=magenta, decoration={coil, amplitude=4pt,segment length=5pt}},
concreto/.style={fill,pattern=north east lines, draw=none, minimum width=0.5cm,minimum height=0.3cm,}
}
\begin{document}
\frame{
\frametitle{Sistema masa-resorte}
\begin{center}
  \begin{animateinline}[loop, poster = first]{25}  
  \whiledo{\thea<135}{
  \begin{tikzpicture}[xscale=0.075,yscale=1,
    declare function={f(\x) = exp(-0.02*\x)*cos(10*\x);}]
    \draw[domain=0:\thea, smooth, variable=\x, cyan] plot (\x, {f(\x)});
    \draw[thick,->,blue] (0,0)--(135,0) node[below] {$t$}; % x axis
    \draw[thick,->,blue] (0,-3)--(0,3) node[left] {$x$}; % y axis
    
%   Crear dibjo del sistema masa resorte
    \coordinate (techo) at (0,2.2);
    \coordinate (orig) at (0,2);
    \coordinate (l) at (0,{f(\thea)});
    \draw[resorte] (orig) -- (l);
    
    \node(wall)[concreto,minimum width=2cm] at (techo){};
    \draw (wall.south east) -- (wall.south west);
    
    \node[red] at (0,{f(\thea)}){$\bullet$};
    \node[red] at (\thea, {f(\thea)}){$\bullet$};
    \draw[draw=black, fill=yellow!30] (-5,{f(\thea)-1}) rectangle (5,{f(\thea)});
  
  \end{tikzpicture}
     \stepcounter{a}
     \ifthenelse{\thea<135}{
    \newframe
     }{
       \end{animateinline}
    }
 }
\end{center}

} 
 
\end{document}
